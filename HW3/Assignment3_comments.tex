
\documentclass{article}

\usepackage{fancyhdr} % Required for custom headers
\usepackage{extramarks} % Required for headers and footers
\usepackage{graphicx} % Required to insert images
\usepackage{enumerate}
\usepackage{amsmath}
\usepackage{bbold}

% Margins
\topmargin=-0.45in
\evensidemargin=0in
\oddsidemargin=0in
\textwidth=6.5in
\textheight=9.0in
\headsep=0.25in 

\linespread{1.1} % Line spacing

% Set up the header and footer
\pagestyle{fancy}
\lhead{Linear Algebra with Application\\
to Engineering Computation}
\chead{}
\rhead{CME 200/ME300A\\
M. Gerritsen\\
Fall 2013}
\headheight = 40pt



%th in the exponent (e.g. when writing ith, instead use i$\eth$)
\newcommand{\eth}{^{\text{th}}}


\newcommand{\R}{\mathbb{R}}

%short-cuts for Greek letters
\newcommand{\al}{\alpha}
\newcommand{\dlt}{\delta}
\newcommand{\eps}{\epsilon}

%times (cross-product)
\newcommand{\x}{\times}
%inverse
\newcommand{\inv}{^{-1}}
%cond
\newcommand{\cond}{\mathrm{cond}}
%trace
\newcommand{\trace}{\mathrm{trace}}

\newcommand{\twith}{\text{ with }}
\newcommand{\tand}{\text{ and }}
\newcommand{\tfor}{\text{ for }}
\newcommand{\tor}{\text{ or }}

\newcommand{\ip}{_{i+1}}
\newcommand{\im}{_{i-1}}

\newcommand{\half}{\frac{1}{2}}
\newcommand{\oneby}[1]{\frac{1}{#1}}
\newcommand{\overto}[1]{\overset{#1}{\longrightarrow}} 
 
\renewcommand\headrulewidth{0.4pt} % Size of the header rule
\renewcommand\footrulewidth{0.4pt} % Size of the footer rule

\setlength\parindent{0pt} % Removes all indentation from paragraphs

%----------------------------------------------------------------------------------------
%	DOCUMENT STRUCTURE COMMANDS
%	Skip this unless you know what you're doing
%----------------------------------------------------------------------------------------

% Header and footer for when a page split occurs within a problem environment
\newcommand{\enterProblemHeader}[1]{
\nobreak\extramarks{#1}{#1 continued on next page\ldots}\nobreak
\nobreak\extramarks{#1 (continued)}{#1 continued on next page\ldots}\nobreak
}

% Header and footer for when a page split occurs between problem environments
\newcommand{\exitProblemHeader}[1]{
\nobreak\extramarks{#1 (continued)}{#1 continued on next page\ldots}\nobreak
\nobreak\extramarks{#1}{}\nobreak
}

\setcounter{secnumdepth}{0} % Removes default section numbers
\newcounter{homeworkProblemCounter} % Creates a counter to keep track of the number of problems

\newcommand{\homeworkProblemName}{}
\newenvironment{homeworkProblem}[1][Problem \arabic{homeworkProblemCounter}]{ % Makes a new environment called homeworkProblem which takes 1 argument (custom name) but the default is "Problem #"
\stepcounter{homeworkProblemCounter} % Increase counter for number of problems
\renewcommand{\homeworkProblemName}{#1} % Assign \homeworkProblemName the name of the problem
\section{\homeworkProblemName} % Make a section in the document with the custom problem count
\enterProblemHeader{\homeworkProblemName} % Header and footer within the environment
}{
\exitProblemHeader{\homeworkProblemName} % Header and footer after the environment
}
\newcommand\overmat[2]{%
  \makebox[0pt][l]{$\smash{\overbrace{\phantom{%
    \begin{matrix}#2\end{matrix}}}^{\text{$#1$}}}$}#2}

\newcommand{\problemAnswer}[1]{ % Defines the problem answer command with the content as the only argument
\noindent\framebox[\columnwidth][c]{\begin{minipage}{0.98\columnwidth}#1\end{minipage}} % Makes the box around the problem answer and puts the content inside
}

\title{Some Comments on Assignment 3}
\date{Issued: October 28, 2013}


%----------------------------------------------------------------------------------------

\begin{document}
\maketitle
\thispagestyle{fancy}

% Problem 1
\begin{homeworkProblem}
\begin{enumerate}[(a)]
\item Basis for column space of a matrix $A$. There are 2 ways to find this. 
\begin{itemize}  
  \item Perform GE on $A$ to get an upper-triangular matrix $U$. Then the basis of column space of $A$ is formed by the columns of ORIGINAL MATRIX $A$ that correspond to the columns with pivots of $U$. Many of you took the columns with the pivots from matrix $U$ which is wrong. 
  \item Perform GE on $A^T$ to get an upper-triangular matrix $U_1$. Then the basis of column space of $A$ is formed by the rows with pivots of $U_1$.
\end{itemize}

\item For this problem, when you get matrix $B$, you have to make sure that $B$ has rank 3. This is because if rank of $B$ is less than $3$, then the null space of $B$ will be strictly larger than the row space of $A$. And if rank of $B$ is larger than $3$, then the null space of $B$ is strictly less than row space of $A$. The problem asks you to find $B$ whose null space is IDENTICAL to row space of $A$. \\ \\
For example
\[ B = \begin{bmatrix} 1 & 0 & 0 & 0 & 0 \\
0 & -2 & 1 & 0 & 0 \end{bmatrix} \]
Note that rank of $B$ is 2. The null space of $B$ is spanned by 
\[ \left\{ \begin{bmatrix} 0 \\ 1 \\ 2 \\ 3 \\ 4 \end{bmatrix}, 
\begin{bmatrix} 0 \\ 0 \\ 0 \\ 1 \\ 2 \end{bmatrix},
\begin{bmatrix} 0 \\ 0 \\ 0 \\ 0 \\ 1 \end{bmatrix}\right\} \]
This is not identical to row space to $A$. 
\end{enumerate}
\end{homeworkProblem}

% Problem 2
\begin{homeworkProblem}
For this problem, you have to show that the conditions on $m, n, r$ that you claim are both {\it sufficient} and {\it necessary} conditions. \\ \\
In order to show that condition $a$ is a {\it sufficient} condition for statement $p$ to be true, you must show that condition $a$ implies statement $p$ ($a \implies p$). \\ \\
In order to show that condition $a$ is a {\it necessary} condition for statement $p$ to be true, you must show that statement $p$ implies condition $a$ ($p \implies a$).
\end{homeworkProblem}

% Problem 5
\begin{homeworkProblem}[Problem 5]
This is a general comment on proving statement $p$ implies statement $q$ ($p \implies q$). There are many ways to do this. However, one thing that is wrong is starting with assume $q$ is true. \\ \\
For example, problem 5 asks you to show that if $V$ and $W$ are 3 dimensional subspaces of $\R^5$, then $V$ and $W$ have a nonzero vector in common. In this case, $p$ is the statement: ``$V$ and $W$ are 3 dimensional subspaces of $\R^5$''. And $q$ is the statement: ``$V$ and $W$ have a nonzero vector in common''. I found that some people start with $q$ is true and try to show that $p$ is true, that is, start with ``suppose $V$ and $W$ have a nonzero vector in common'', and try to show that ``$V$ and $W$ are 3 dimensional subspaces of $\R^5$''. This is not correct.   
\end{homeworkProblem}

\end{document}
 
 
 
