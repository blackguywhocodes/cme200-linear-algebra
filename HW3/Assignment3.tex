
\documentclass{article}

\usepackage{fancyhdr} % Required for custom headers
\usepackage{extramarks} % Required for headers and footers
\usepackage{graphicx} % Required to insert images
\usepackage{enumerate}
\usepackage{amsmath}
\usepackage{bbold}

% Margins
\topmargin=-0.45in
\evensidemargin=0in
\oddsidemargin=0in
\textwidth=6.5in
\textheight=9.0in
\headsep=0.25in 

\linespread{1.1} % Line spacing

% Set up the header and footer
\pagestyle{fancy}
\lhead{Linear Algebra with Application\\
to Engineering Computation}
\chead{}
\rhead{CME 200/ME300A\\
M. Gerritsen\\
Fall 2013}
\headheight = 40pt



%th in the exponent (e.g. when writing ith, instead use i$\eth$)
\newcommand{\eth}{^{\text{th}}}


\newcommand{\R}{\mathbb{R}}

%short-cuts for Greek letters
\newcommand{\al}{\alpha}
\newcommand{\dlt}{\delta}
\newcommand{\eps}{\epsilon}

%times (cross-product)
\newcommand{\x}{\times}
%inverse
\newcommand{\inv}{^{-1}}
%cond
\newcommand{\cond}{\mathrm{cond}}
%trace
\newcommand{\trace}{\mathrm{trace}}

\newcommand{\twith}{\text{ with }}
\newcommand{\tand}{\text{ and }}
\newcommand{\tfor}{\text{ for }}
\newcommand{\tor}{\text{ or }}

\newcommand{\ip}{_{i+1}}
\newcommand{\im}{_{i-1}}

\newcommand{\half}{\frac{1}{2}}
\newcommand{\oneby}[1]{\frac{1}{#1}}
\newcommand{\overto}[1]{\overset{#1}{\longrightarrow}} 
 
\renewcommand\headrulewidth{0.4pt} % Size of the header rule
\renewcommand\footrulewidth{0.4pt} % Size of the footer rule

\setlength\parindent{0pt} % Removes all indentation from paragraphs

%----------------------------------------------------------------------------------------
%	DOCUMENT STRUCTURE COMMANDS
%	Skip this unless you know what you're doing
%----------------------------------------------------------------------------------------

% Header and footer for when a page split occurs within a problem environment
\newcommand{\enterProblemHeader}[1]{
\nobreak\extramarks{#1}{#1 continued on next page\ldots}\nobreak
\nobreak\extramarks{#1 (continued)}{#1 continued on next page\ldots}\nobreak
}

% Header and footer for when a page split occurs between problem environments
\newcommand{\exitProblemHeader}[1]{
\nobreak\extramarks{#1 (continued)}{#1 continued on next page\ldots}\nobreak
\nobreak\extramarks{#1}{}\nobreak
}

\setcounter{secnumdepth}{0} % Removes default section numbers
\newcounter{homeworkProblemCounter} % Creates a counter to keep track of the number of problems

\newcommand{\homeworkProblemName}{}
\newenvironment{homeworkProblem}[1][Problem \arabic{homeworkProblemCounter}]{ % Makes a new environment called homeworkProblem which takes 1 argument (custom name) but the default is "Problem #"
\stepcounter{homeworkProblemCounter} % Increase counter for number of problems
\renewcommand{\homeworkProblemName}{#1} % Assign \homeworkProblemName the name of the problem
\section{\homeworkProblemName} % Make a section in the document with the custom problem count
\enterProblemHeader{\homeworkProblemName} % Header and footer within the environment
}{
\exitProblemHeader{\homeworkProblemName} % Header and footer after the environment
}
\newcommand\overmat[2]{%
  \makebox[0pt][l]{$\smash{\overbrace{\phantom{%
    \begin{matrix}#2\end{matrix}}}^{\text{$#1$}}}$}#2}

\newcommand{\problemAnswer}[1]{ % Defines the problem answer command with the content as the only argument
\noindent\framebox[\columnwidth][c]{\begin{minipage}{0.98\columnwidth}#1\end{minipage}} % Makes the box around the problem answer and puts the content inside
}

\title{Assignment 3}
\date{Issued: October 9, 2013}
\author{Due: October 16, in class\\
No late assignments accepted}

%----------------------------------------------------------------------------------------

\begin{document}
\maketitle
\thispagestyle{fancy}
\textbf{Important:}
\begin{itemize}
\item Give complete answers: Do not only give mathematical formulae, but explain what you are doing. Conversely, do not leave out critical intermediate steps in mathematical derivations.
\item Write your \textbf{name} as well as your \textbf{Sunet ID} on your assignment. \textbf{Please staple pages together.} Points will be docked otherwise.
\item Questions preceded by  $\star$  are harder and/or more involved.
\end{itemize}


% Problem 1
\begin{homeworkProblem}
\begin{enumerate}[(a)]
%(5)
\item Find a basis for the column space and row space of matrix $A$ given by
\[ A = \begin{bmatrix}
0 & 1 & 2 & 3 & 4 \\
0 & 1 & 2 & 4 & 6 \\
0 & 0 & 0 & 1 & 2
\end{bmatrix} \]

 %(10)
\item  Construct a matrix $B$ such that the null space of $B$ is identical to the row space of $A$.
\end{enumerate}
\end{homeworkProblem}

% Problem 2
\begin{homeworkProblem} %(15)
Let $A$ be an $m \x n$ matrix with rank $r \leq \min\{m,n\}$. Depending on $m,\ n \tand r$, a system $A\vec x=\vec b$ can have none, one, or infinitely many solutions.\\
For what choices of $m,\ n \tand r$ do each of the following cases hold? If no such $m,\ n \tand r$ can be found explain why not.
\begin{enumerate}[(a)]
\item $A\vec{x}=\vec{b}$ has no solutions, regardless of $\vec{b}$
\item $A\vec{x}=\vec{b}$ has exactly 1 solution for any $\vec{b}$
\item $A\vec{x}=\vec{b}$ has infinitely many solutions for any $\vec{b}$
%\item $A\vec{x}=\vec{b}$ has no {\bf or} just 1 solution depending on $\vec{b}$
%\item $A\vec{x}=\vec{b}$ has no {\bf or} infinitely many solutions depending on $\vec{b}$
%\item $A\vec{x}=\vec{b}$ has only 1 {\bf or} infinitely many solutions depending on $\vec{b}$
\end{enumerate}
({\bf Hint}: think of what conditions the column vectors and column space of $A$ should satisfy.)
\end{homeworkProblem}
  
  
% Problem 3
\begin{homeworkProblem} %(15)
Consider a matrix product $AB$, where $A$ is $m\x n \tand B$ is $n\x p$. Show that the column space of $AB$ is contained in the column space of $A$. Give an example of matrices $A,\ B$ such that those two spaces are not identical.\\
{\bf Definition.} A vector space $U$ is {\it contained} in another vector space $V$ (denoted as $U \subseteq V$) if every vector $\vec u \in U$ ($\vec u$ in vector space $U$) is also in $V$.
\\
{\bf Definition.} We say that two vector spaces are {\it identical} (equal) if $U \subseteq V$ {\bf and} $V \subseteq U$. \\
(e.g. $V$ is identical to itself since $V \subseteq V$ and $V \subseteq V$.)
\end{homeworkProblem}
  

% Problem 4
\begin{homeworkProblem} %(15)
An $n \x n$ matrix $A$ has a property that the elements in each of its rows sum to 1. Let $P$ be any $n \x n$ permutation matrix. Prove that $(P-A)$ is singular.
\end{homeworkProblem}
  

% Problem 5
\begin{homeworkProblem} %(15)
Let $V \tand W$ be 3 dimensional subspaces of $\R^5$. Show that $V \tand W$ must have at least one nonzero vector in common.
\end{homeworkProblem}


% Problem 6
\begin{homeworkProblem}
\begin{enumerate}[(a)]
%(5)
\item The nonzero column vectors $\vec{u} \tand \vec{v}$ have $n$ elements. An $n \x n$ matrix $A$ is given by $A = \vec{u}\vec{v}^T$ ({\bf Note:} this is different from the innerproduct (also sometimes known as the dot product), which we would write as $\vec{v}^T\vec{u}$). Show that the rank of $A$ is 1.

 %(10)
\item Show that the converse is true. That is, if the rank of a matrix $A$ is 1, then we
can find two vectors $\vec{u} \tand \vec{v}$, such that $A = \vec{u}\vec{v}^T$.
\end{enumerate}
\end{homeworkProblem}

\end{document}
 
 
 
