
\documentclass{article}

\usepackage{amsmath}
\usepackage{amssymb}
\usepackage{bbm}
\usepackage{comment}
\usepackage{enumerate}

\usepackage{extramarks} % Required for headers and footers
\usepackage{fancyhdr}

\usepackage{graphicx} % Required to insert images
\usepackage{verbatim}


% Margins
\topmargin=-0.45in
\evensidemargin=0in
\oddsidemargin=0in
\textwidth=6.5in
\textheight=9.0in
\headsep=0.25in 

\linespread{1.1} % Line spacing

% Set up the header and footer
\pagestyle{fancy}
\lhead{Linear Algebra with Application\\
to Engineering Computation}
\chead{}
\rhead{CME 200/ME300A\\
M. Gerritsen\\
Fall 2013}
\headheight = 40pt

\renewcommand\headrulewidth{0.4pt} % Size of the header rule
\renewcommand\footrulewidth{0.4pt} % Size of the footer rule

\setlength\parindent{0pt} % Removes all indentation from paragraphs

%Add useful short-cut commands here

%math symbols such as R, and Pr
\newcommand{\I}{\ensuremath{\operatorname{I}}}
\newcommand{\One}[1]{\ensuremath{\mathbbm{1}_{\left \{ #1 \right \}}}}
\newcommand{\E}{\ensuremath{\mathbb{E}}}
\newcommand{\R}{\ensuremath{\mathbb{R}}}
\newcommand{\N}{\ensuremath{\mathbb{N}}}


%1st order derivative wrt x
\newcommand{\dxone}[1]{\frac{d#1}{dx}}
%2nd order derivative wrt x
\newcommand{\dxtwo}[1]{\frac{d^2#1}{dx^2}}
%th in the exponent (e.g. when writing ith, instead use i$\eth$)
\newcommand{\tth}{^{\text{th}}}

%short-cuts for Greek letters
\newcommand{\al}{\alpha}
\newcommand{\dlt}{\delta}
\newcommand{\eps}{\epsilon}
\newcommand{\la}{\lambda}


%norms
\newcommand{\norm}[1]{\|#1\|}
\newcommand{\normII}[1]{\|#1\|_2}

\newcommand{\x}{\times}
%inverse
\newcommand{\inv}{^{-1}}
%cond
\newcommand{\cond}{\mathrm{cond}}
%trace
\newcommand{\trace}{\mathrm{trace}}
\newcommand{\tr}{\mathrm{tr}}
%rank
\newcommand{\rank}{\mathrm{rank}}

 
%matrices
\newcommand{\bmat}[1]{\begin{bmatrix}#1\end{bmatrix}} 
\newcommand{\pmat}[1]{\begin{pmatrix}#1\end{pmatrix}} 
 

%parentheses
\newcommand{\paren}[1]{\left(#1\right)}
\newcommand{\brac}[1]{\left[#1\right]}
\newcommand{\cbrac}[1]{\left\{#1\right\}} 


\newcommand{\twith}{\text{ with }}
\newcommand{\tand}{\text{ and }}
\newcommand{\tfor}{\text{ for }}
\newcommand{\tor}{\text{ or }}
\newcommand{\tat}{\text{ at }}

\newcommand{\ip}{_{i+1}}
\newcommand{\im}{_{i-1}}

\newcommand{\half}{\frac{1}{2}}
\newcommand{\oneby}[1]{\frac{1}{#1}}
\newcommand{\overto}[1]{\overset{#1}{\longrightarrow}}
%----------------------------------------------------------------------------------------
%       DOCUMENT STRUCTURE COMMANDS
%       Skip this unless you know what you're doing
%----------------------------------------------------------------------------------------

% Header and footer for when a page split occurs within a problem environment
\newcommand{\enterProblemHeader}[1]{
\nobreak\extramarks{#1}{#1 continued on next page\ldots}\nobreak
\nobreak\extramarks{#1 (continued)}{#1 continued on next page\ldots}\nobreak
}

% Header and footer for when a page split occurs between problem environments
\newcommand{\exitProblemHeader}[1]{
\nobreak\extramarks{#1 (continued)}{#1 continued on next page\ldots}\nobreak
\nobreak\extramarks{#1}{}\nobreak
}

\setcounter{secnumdepth}{0} % Removes default section numbers
\newcounter{homeworkProblemCounter} % Creates a counter to keep track of the number of problems

\newcommand{\homeworkProblemName}{}
\newenvironment{homeworkProblem}[1][Problem \arabic{homeworkProblemCounter}]{ % Makes a new environment called homeworkProblem which takes 1 argument (custom name) but the default is "Problem #"
\stepcounter{homeworkProblemCounter} % Increase counter for number of problems
\renewcommand{\homeworkProblemName}{#1} % Assign \homeworkProblemName the name of the problem
\section{\homeworkProblemName} % Make a section in the document with the custom problem count
\enterProblemHeader{\homeworkProblemName} % Header and footer within the environment
}{
\exitProblemHeader{\homeworkProblemName} % Header and footer after the environment
}
\newcommand\overmat[2]{%
  \makebox[0pt][l]{$\smash{\overbrace{\phantom{%
    \begin{matrix}#2\end{matrix}}}^{\text{$#1$}}}$}#2}

\newcommand{\problemAnswer}[1]{ % Defines the problem answer command with the content as the only argument
\noindent\framebox[\columnwidth][c]{\begin{minipage}{0.98\columnwidth}#1\end{minipage}} % Makes the box around the problem answer and puts the content inside
}

\title{Assignment 7 - Solutions}
\date{Issued: November 13, 2013}
\author{Due: November 20, in class\\
No late assignments accepted}

%----------------------------------------------------------------------------------------

\begin{document}

\maketitle
\thispagestyle{fancy}

% Problem 1
\begin{homeworkProblem}
For this problem we assume that eigenvalues and eigenvectors are all real valued.
\begin{enumerate}[(a)]

\item Let $A$ be an $n \x n$ symmetric matrix.
Let $\vec q_i$ and $\vec q_j$ be the eigenvectors of $A$ corresponding to the eigenvalues $\la_i$ and $\la_j$ respectively. Show that if $\la_i \ne \la_j$, then $\vec q_i$ and $\vec q_j$ are orthogonal.

\item Let $A$ be an $n \x n$ matrix. We say that $A$ is {\bf positive definite} if for any non-zero vector $\vec x$, the following inequality holds
\[ \vec x^T A \vec x > 0. \]
Show that the eigenvalues of a positive definite matrix $A$ are all positive.

\item$\star\star$ Let $A$ be an $n \x n$ matrix. Show that 
\[ \tr(A) = \sum_{i=1}^n \la_i, \] 
where $\la_1, \dots, \la_n$ are the eigenvalues of $A$ ($\la_i$'s do not have to be all different).

[Hint 1: One way to prove this is to use the fact that any square matrix with real eigenvalues can be decomposed in the following way (called Schur decomposition)
\[A = QRQ^T,\]
where $R$ is an upper triangular matrix and $Q$ is an orthogonal matrix.]

[Hint 2: The following property of trace might be useful: given two matrices $A \in \R^{m \x n}\ \tand B \in \R^{n \x m}$, the trace of their product, $\tr(AB)$, is \textit{invariant under cyclic permutations}, i.e. $\tr(AB) = \tr(BA)$.\\
Note that this implies $\tr(ABC) = \tr(BCA) = \tr(CAB)$ for any matrices $A,\ B,\ C$ with appropriately chosen dimension.]
\end{enumerate}
\end{homeworkProblem}

{\bf Solution:}

\begin{enumerate}[(a)]
\item By the definition of eigenvalues/eigenvectors, we have
\[ A\vec q_i = \la_i \vec q_i, \quad A\vec q_j = \la_j \vec q_j \]
Then, since $A = A^T$, 
\[ \la_i \vec q_j^T \vec q_i = \vec q_j^T \la_i \vec q_i = \vec q_j^T A \vec q_i = \vec q_j^T A^T \vec q_i = (A \vec q_j)^T \vec q_i = (\la_j \vec q_j)^T \vec q_i = \la_j \vec q_j^T \vec q_i. \]
Rearranging, we have
\[ (\la_i - \la_j) \vec q_j^T \vec q_i = 0. \]
When $\la_i \ne \la_j$ the term in the parentheses is non-zero, so it follows that
\[ \vec q_j^T \vec q_i = 0, \]
which means that $\vec q_i$ and $\vec q_j$ are orthogonal.

\item Let $\la$ be an eigenvalue of $A$ corresponding to an eigenvector $\vec q$, i.e.
\[ A\vec q = \la \vec q. \]
Then, since $A$ is positive definite, we have
\[ 0 < \vec q^T A \vec q = \vec q^T \la \vec q = \la \normII{\vec q}^2. \]
Since $\normII{\vec q} > 0$ ($\vec q \ne \vec 0$ for any eigenvector $\vec q$), it follows that $\la>0$.

\item Consider the Schur decomposition of $A$,
\[ A = QRQ^T. \]
Using the property of trace that $\tr(ABC) = \tr(BCA) = \tr(CAB)$ (invariance under cyclic permutations), we have
\[ \tr(A) = \tr(QRQ^T) = \tr(RQ^TQ) = \tr(R) = \sum_{i=1}^n R_{ii}. \]
We now prove that for any (upper) triangular matrix $R$ the diagonal elements of $R$ are its eigenvalues. Consider the characteristic polynomial of $R$,
\[ p(\la) = \det(\la I - R) = (\la-R_{11})(\la-R_{22})\cdots(\la-R_{nn}) = \prod_{i=1}^n (\la-R_{ii}), \]
where for the second equality we used the fact that the determinant of a triangular matrix is the product of its diagonal entries.
Since the roots of a characteristic equation are the eigenvalues of $R$, we can immediately see that $\la_i = R_{ii}\tfor i=1,\dots, n$, so that
\[ \tr(A) = \sum_{i=1}^n R_{ii} = \sum_{i=1}^n \la_i. \]
Now, it remains to show that the eigenvalues of $R$ are equal to the eigenvalues of $A$.
If $\al$ is an eigenvalue of $A$ corresponding to eigenvector $\vec v$, then
\[ \al \vec v = A \vec v = QRQ^T \vec v. \]
Pre-multiplying by $Q^T$, we have
\[ Q^T \al \vec v = RQ^T \vec v, \]
\[ \al Q^T \vec v = RQ^T \vec v, \]
\[ \al \vec u = R \vec u, \]
where $\vec u = Q^T \vec v$ is an eigenvector of $R$ corresponding to eigenvalue $\al$.\\ 
Similarly, if $\beta$ is an eigenvalue of $R$ corresponding to eigenvector $\vec u$, then
\[ \beta \vec u = R \vec u = Q^TAQ \vec u. \]
Pre-multiplying by $Q$, we have
\[ \beta Q \vec u = AQ \vec u, \]
\[ \beta \vec v = A \vec v, \]
where $\vec v = Q \vec u$ is an eigenvector of $A$ corresponding to eigenvalue $\beta$.\\
This proves the claim. Therefore,
\[ \tr(A) = \sum_{i=1}^n \la_i, \]
where $\la_i,\ i=1,\dots, n$ are the eigenvalues of $A$ (and $R$).

[{\bf Note.} The fact that eigenvalues of $A$ and $R$ were the same holds in a more general setting. We say that two matrices $A$ and $B$ are {\bf similar} if there exists and invertible matrix $P$, such that 
\[ B = P\inv A P. \]
The proof above shows that all similar matrices $A$ and $B$ have the same set of eigenvalues. This property is referred to as the {\bf invariance of eigenvalues under similarity transformations}.]
\end{enumerate}



% Problem 2
\begin{homeworkProblem} %(15)

We are interested in finding the fixed points (the points at which the time derivatives are zero) of the following system of equations:
\begin{align*}
\frac{dx_1}{dt} &= x_1(a-bx_2)\\
\frac{dx_2}{dt} &= -x_2(c-dx_1)
\end{align*}
for $a = 3, b = 1, c = 2, d = 1$. We can use the Newton-Raphson method to find these fixed points, simply by setting the derivatives zero in the given system of equations. 
\begin{enumerate}[(a)]
\item
In the scalar case, Newton-Raphson breaks down at points at which the derivative of the nonlinear function is zero. In general, where can it break down for systems of nonlinear equations? For the system given above, find the troublesome points.
\item
Find the fixed points of the above system analytically. 
\item
Find all fixed points using repeated application of the Newton-Raphson method. You will have to judiciously choose your starting points (but of course, you are not allowed to use the known roots as starting points!). You may use MATLAB to program the method if you like. 
\end{enumerate}
{\bf Solution:}
\begin{enumerate}
\item
The Newton-Raphson method can break down for systems of nonlinear equations when the Jacobian matrix is singular. Let’s examine the given system:
\begin{align*}
f_1(x_1,x_2) &= x_1(3-x_2)\\
f_2(x_1,x_2) &= -x_2(2-x_1)\\
J(x_1,x_2) &= \pmat{\frac{\partial f_1}{\partial x_1} & \frac{\partial f_1}{\partial x_2}\\\frac{\partial f_2}{\partial x_1} & \frac{\partial f_2}{\partial x_2}} \\
&= \pmat{3-x_2 & -x_1 \\ x_2 & x_1 - 2}
\end{align*}
The troublesome points can be found by setting the determinant of $J=0$
\begin{align*}
\det(J) &= (3-x_2)(x_1-2)+x_1x_2\\
&= 0\\
3x_1+2x_2 &= 0
\end{align*}
Hence, the troublesome points in this case form a line in two dimensional space.
\item
The fixed points of the above system occur when the derivatives of $x$ w.r.t. $t$ are zero. Therefore
\begin{align*}
f_1(x_1,x_2) &= x_1(3-x_2) = 0\\
f_2(x_1,x_2) &= -x_2(2-x_1) = 0\\
\end{align*}
This is easily solved to be (0,0) and (2,3).
\item
The equation to solve at every iteration is
\begin{align*}
J(\vec{x}^{(k)})(\vec{x}^{(k+1)}-\vec{x}^{(k)}) 
= -f(\vec{x}^{(k)})
\end{align*}
\begin{align*}
\pmat{{x_1}^{(k+1)}\\ {x_2}^{(k+1)}} = -\pmat{3-{x_2}^{(k)}  & {x_1}^{(k)}\\{x_2}^{(k)} & {x_1}^{(k)} - 2}^{-1}\pmat{x_1^{(k)}(3-x_2^{(k)}) \\ x_2^{(k)}(2-x_1^{(k)})} + \pmat{x_1^{(k)} \\x_2^{(k)}}
\end{align*}
This iteration can be computed in Matlab or by hand. Of course, we should not come across points that make the Jacobian singular and therefore, we choose the starting points to be not close to the line of troublesome points calculated in the first part. Interestingly, travelling across the line when choosing the starting point, changes the final solution. So, we choose one starting point which lies above the line(e.g. (4,2)) to get (2,3) and another which is below the line(e.g (-1,1)) to get the other solution(0,0). This takes about 5 iterations to converge.
\begin{align*}
\vec{x}^{(1)} = \pmat{4\\2} \rightarrow 
\vec{x}^{(2)} = \pmat{1.6\\2.4} \rightarrow 
\vec{x}^{(3)} = \pmat{2.1333\\3.2}\rightarrow 
\vec{x}^{(4)} = \pmat{2.0078\\3.2}\rightarrow 
\vec{x}^{(5)} = \pmat{2\\3}
\end{align*}
\begin{align*}
\vec{x}^{(1)} = \pmat{-1\\1} \rightarrow 
\vec{x}^{(2)} = \pmat{0.2857\\0.4286} \rightarrow 
\vec{x}^{(3)} = \pmat{-0.0015\\-0.0023}\rightarrow 
\vec{x}^{(4)} = 10^{-5}\pmat{0.1191\\0.1786}\rightarrow 
\vec{x}^{(5)} = 10^{-11}\pmat{-0.0709\\-0.1063}
\end{align*}

\end{enumerate}
\end{homeworkProblem}

% Problem 3
\begin{homeworkProblem}
\begin{enumerate}[(a)]
\item
If $P^2 = P$, show that 
\begin{align*}
e^P \approx I + 1.718P
\end{align*}
\item
Convert the equation below to a matrix equation and then by using the exponential matrix find the solution in terms of $y(0)$ and $y'(0)$:
\begin{align*}
y'' = 0
\end{align*} 
\item
Show that $e^{A+B}=e^Ae^B$ is not generally true for matrices. 
\end{enumerate}

{\bf Solution:}
\begin{enumerate}[(a)]
\item
From the definition of exponential matrices we have:
\begin{align*}
e^P &= I + P + \frac{P^2}{2!} + \frac{P^3}{3!} + \cdots
\end{align*}
In addition we know that:
\begin{align*}
P^2 &= P\\
\Rightarrow P^n &= P \text{\quad $\forall n$ by simple induction}
\end{align*}
Therefore
\begin{align*}
e^P &= I + P + \frac{P^2}{2!} + \frac{P^3}{3!} + \cdots\\
&= I + P + \frac{P}{2!} + \frac{P}{3!} + \cdots \\
&= I + P(1 + \frac{1}{2!} + \frac{1}{3!} + \cdots) \\
\end{align*}
From Taylor expansion of exponential function we have
\begin{align*}
e^x &= 1 + \frac{x}{1!} + \frac{x^2}{2!} + \frac{x^3}{3!} \cdots\\
e &= 1 + \frac{1}{1!} + \frac{1}{2!} + \frac{1}{3!} \cdots\\
e - 1 &= 1 + \frac{1}{2!} + \frac{1}{3!} \cdots\\
&= 1.718
\end{align*}
Therefore 
\begin{align*}
e^P &= I + P(1 + \frac{1}{2!} + \frac{1}{3!}\cdots) \\
&\approx I + 1.718P\\
\end{align*}
\item
In order to transform this ODE to a matrix equation, we use:
\begin{align*}
\frac{d}{dt}\pmat{y\\ y'} &= \pmat{y' \\ y''}\\
&= \pmat{y' \\ 0}\\
&= \pmat{0 & 1 \\ 0 & 0}\pmat{y\\ y'}\\
\end{align*}
Therefore taking $A = \pmat{0 & 1 \\ 0 & 0}$
\begin{align*}
\frac{d}{dt}\pmat{y\\ y'} &= A \pmat{y\\ y'}\\
\Rightarrow \pmat{y\\ y'} &= e^{At} \pmat{y(0)\\ y'(0)}
\end{align*}
Now we look at
\begin{align*}
e^{At} &= I + At + \frac{(At)^2}{2!} + \frac{(At)^3}{3!}\cdots\\
&= I + At + \frac{A^2}{2!}t^2 + \frac{A^3}{3!}t^3\cdots
\end{align*}
But $A$ is nilpotent, i.e.
\begin{align*}
A^2 &= \pmat{0 & 1 \\ 0 & 0}\pmat{0 & 1 \\ 0 & 0}\\
&= \pmat{0 & 0 \\ 0 & 0}\\
&= A^n \text{\quad $\forall n$ by simple induction}
\end{align*}
Therefore 
\begin{align*}
e^{At} &= I + At \\
&= \pmat{1 & t \\ 0 & 1}
\end{align*}
\begin{align*}
\pmat{y \\ y'} &= \pmat{y(0) \\ y'(0)}e^{At}\\
&= \pmat{1 & t \\ 0 & 1}\pmat{y(0) \\ y'(0)}\\
&= \pmat{y(0)+y'(0)t\\y'(0)}
\end{align*}
Therefore we have $y = y(0)+y'(0)t$.

\item
Let
\begin{align*}
A = \pmat{0 & 0 \\ 1 & 0}, \quad B &= \pmat{0 & -1\\0 & 0}\\
\Rightarrow A^2 &= B^2 = 0\\
\Rightarrow e^A &= I + A\\
 e^B &= I + B\\
\end{align*}
Therefore
\begin{align*}
e^Ae^B &= I + A + B + AB\\
e^Be^A &= I + B + A + BA
\end{align*}
\begin{align*}
AB = \pmat{0 & 0 \\ 0 & -1}, BA = \pmat{0 & 0\\ -1 & 0}
\end{align*}
Therefore $e^Ae^B \neq e^Be^A$ for these two matrices.
Now suppose for contradiction that $e^{A+B}=e^Ae^B$. Then
\begin{align*}
e^Ae^B 
&= e^{A+B}\\
&= e^{B+A} \text{\quad addition is always commutative}\\
&= e^Be^A
\end{align*}
This is clearly not true because we showed earlier that $e^Ae^B \neq e^Be^A$. Therefore $e^Ae^B = e^{A+B}$ is generally not true.

\end{enumerate}
\end{homeworkProblem}


\end{document}
