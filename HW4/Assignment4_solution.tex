%

\documentclass{article}

\usepackage{fancyhdr} % Required for custom headers
\usepackage{extramarks} % Required for headers and footers
\usepackage{graphicx} % Required to insert images
\usepackage{enumerate}
\usepackage{amsmath}
\usepackage{bbold}

% Margins
\topmargin=-0.45in
\evensidemargin=0in
\oddsidemargin=0in
\textwidth=6.5in
\textheight=9.0in
\headsep=0.25in 

\linespread{1.1} % Line spacing

% Set up the header and footer
\pagestyle{fancy}
\lhead{Linear Algebra with Application\\
to Engineering Computation}
\chead{}
\rhead{CME 200/ME300A\\
M. Gerritsen\\
Fall 2013}
\headheight = 40pt



%th in the exponent (e.g. when writing ith, instead use i$\eth$)
\newcommand{\eth}{^{\text{th}}}


\newcommand{\R}{\mathbb{R}}

%short-cuts for Greek letters
\newcommand{\al}{\alpha}
\newcommand{\dlt}{\delta}
\newcommand{\eps}{\epsilon}

%times (cross-product)
\newcommand{\x}{\times}
%inverse
\newcommand{\inv}{^{-1}}
%cond
\newcommand{\cond}{\mathrm{cond}}
%trace
\newcommand{\trace}{\mathrm{trace}}

\newcommand{\twith}{\text{ with }}
\newcommand{\tand}{\text{ and }}
\newcommand{\tfor}{\text{ for }}
\newcommand{\tor}{\text{ or }}

\newcommand{\ip}{_{i+1}}
\newcommand{\im}{_{i-1}}

\newcommand{\half}{\frac{1}{2}}
\newcommand{\oneby}[1]{\frac{1}{#1}}
\newcommand{\overto}[1]{\overset{#1}{\longrightarrow}} 
 
\renewcommand\headrulewidth{0.4pt} % Size of the header rule
\renewcommand\footrulewidth{0.4pt} % Size of the footer rule

\setlength\parindent{0pt} % Removes all indentation from paragraphs

%----------------------------------------------------------------------------------------
%       DOCUMENT STRUCTURE COMMANDS
%       Skip this unless you know what you're doing
%----------------------------------------------------------------------------------------

% Header and footer for when a page split occurs within a problem environment
\newcommand{\enterProblemHeader}[1]{
\nobreak\extramarks{#1}{#1 continued on next page\ldots}\nobreak
\nobreak\extramarks{#1 (continued)}{#1 continued on next page\ldots}\nobreak
}

% Header and footer for when a page split occurs between problem environments
\newcommand{\exitProblemHeader}[1]{
\nobreak\extramarks{#1 (continued)}{#1 continued on next page\ldots}\nobreak
\nobreak\extramarks{#1}{}\nobreak
}

\setcounter{secnumdepth}{0} % Removes default section numbers
\newcounter{homeworkProblemCounter} % Creates a counter to keep track of the number of problems

\newcommand{\homeworkProblemName}{}
\newenvironment{homeworkProblem}[1][Problem \arabic{homeworkProblemCounter}]{ % Makes a new environment called homeworkProblem which takes 1 argument (custom name) but the default is "Problem #"
\stepcounter{homeworkProblemCounter} % Increase counter for number of problems
\renewcommand{\homeworkProblemName}{#1} % Assign \homeworkProblemName the name of the problem
\section{\homeworkProblemName} % Make a section in the document with the custom problem count
\enterProblemHeader{\homeworkProblemName} % Header and footer within the environment
}{
\exitProblemHeader{\homeworkProblemName} % Header and footer after the environment
}
\newcommand\overmat[2]{%
  \makebox[0pt][l]{$\smash{\overbrace{\phantom{%
    \begin{matrix}#2\end{matrix}}}^{\text{$#1$}}}$}#2}

\newcommand{\problemAnswer}[1]{ % Defines the problem answer command with the content as the only argument
\noindent\framebox[\columnwidth][c]{\begin{minipage}{0.98\columnwidth}#1\end{minipage}} % Makes the box around the problem answer and puts the content inside
}

\title{Assignment 4 - Solutions}
\date{Issued: October 16, 2013}
\author{Due: October 23, in class\\
No late assignments accepted}

%----------------------------------------------------------------------------------------

\begin{document}

\maketitle
\thispagestyle{fancy}
\textbf{Important:}
\begin{itemize}
\item Give complete answers: Do not only give mathematical formulae, but explain what you are doing. Conversely, do not leave out critical intermediate steps in mathematical derivations.
\item Write your \textbf{name} as well as your \textbf{Sunet ID} on your assignment. \textbf{Please staple pages together.}
\item Questions preceded by  $\star$  are harder and/or more involved.
\item \textbf{Include code with your assignment.}
\item Comment any graphs and plots on the same page as the graph or plot itself.
\end{itemize}

% Problem 1
\begin{homeworkProblem} %(15)
In assignment 2, we discretized the 1-dimensional heat equation with Dirichlet boundary conditions:
\begin{align*}
\frac{d^2T}{dx^2} &= 0,0\leq x\leq 1 \\
T(0) &= 0,T(1) = 2
\end{align*}
The discretization leads to the matrix-vector equation $At = b$, with
\begin{align*}
A = 
\begin{pmatrix}
-2 &  1  & 0 & \ldots & 0\\
1  &  -2 & 1 & \ldots & 0\\
\vdots & \ddots & \ddots & \ddots & \vdots\\
0   &\ldots & 1 & -2 & 1\\
0   &\ldots & 0 & 1 & -2
\end{pmatrix}, b =
\begin{pmatrix}
0\\
0\\
\vdots\\
0 \\
-2
\end{pmatrix}
\end{align*}
Here $A$ is an $(N-1) \times (N-1)$ matrix.
\begin{enumerate}[a.]
\item
(10 pts) Find the LU factorization of $A$ for $N = 10$ using \texttt{Matlab}. Is \texttt{Matlab} using any pivoting to find the LU decomposition? Find the inverse
of $A$ also. As you can see the inverse of $A$ is a dense matrix.
\textbf{Note:} The attractive sparsity of $A$ has been lost when computing its inverse, but $L$ and $U$ are sparse. Generally speaking, banded matrices have $L$ and $U$ with similar band structure. Naturally we then prefer to use the $L$ and $U$ matrices to compute the solution, and not the inverse. Finding $L$ and $U$ matrices with least "fill-in" ("fill-in" refers to nonzeros appearing at locations in the matrix where $A$ has a zero element) is an active research area, and generally involves sophisticated matrix re-ordering algorithms.

\textbf{Solution:}

The solution can be found using Matlab command \texttt{lu}. To check whether Matlab is using any pivoting we can see what permutation matrix $P$ is returned by the \texttt{lu} command. We can see it if we use \texttt{[L, U, P] = lu(A)} as in the code given below. For the above matrix, with $N=10$ we get $L, U, P$ as follows: 
\begin{align*}
L = 
\begin{pmatrix}
1  & 0 & 0 & 0 & 0 & 0 & 0 & 0 & 0 \\
-1/2  & 1 & 0 & 0 & 0 & 0 & 0 & 0 & 0 \\
0  & -2/3 & 1 & 0 & 0 & 0 & 0 & 0 & 0 \\
0  & 0 & -3/4 & 1 & 0 & 0 & 0 & 0 & 0 \\
0  & 0 & 0 & -4/5 & 1 & 0 & 0 & 0 & 0 \\
0  & 0 & 0 & 0 & -5/6 & 1 & 0 & 0 & 0 \\
0  & 0 & 0 & 0 & 0 & -6/7 & 1 & 0 & 0 \\
0  & 0 & 0 & 0 & 0 & 0 & -7/8 & 1 & 0 \\
0  & 0 & 0 & 0 & 0 & 0 & 0 & -8/9 & 1 \\
\end{pmatrix}
\end{align*}
\begin{align*}
U = 
\begin{pmatrix}
-2  & 1 & 0 & 0 & 0 & 0 & 0 & 0 & 0 \\
0  & -3/2 & 1 & 0 & 0 & 0 & 0 & 0 & 0 \\
0  & 0 & -4/3 & 1 & 0 & 0 & 0 & 0 & 0 \\
0  & 0 & 0 & -5/4 & 1 & 0 & 0 & 0 & 0 \\
0  & 0 & 0 & 0 & -6/5 & 1 & 0 & 0 & 0 \\
0  & 0 & 0 & 0 & 0 & -7/6 & 1 & 0 & 0 \\
0  & 0 & 0 & 0 & 0 & 0 & -8/7 & 1 & 0 \\
0  & 0 & 0 & 0 & 0 & 0 & 0 & -9/8 & 1 \\
0  & 0 & 0 & 0 & 0 & 0 & 0 & 0 & -10/9 \\
\end{pmatrix}
\end{align*}

and $P$ is the identity matrix. Hence we know that \texttt{Matlab} is not using any pivoting in this case. 
While $L$ and $U$ maintain the sparsity of $A$, the $A\inv$ does not.

\begin{align*}
A\inv = 
\begin{pmatrix}
0.9  & 0.8 & 0.7 & 0.6 & 0.5 & 0.4 & 0.3 & 0.2 & 0.1 \\
0.8  & 1.6 & 1.4 & 1.2 & 1.0 & 0.8 & 0.6 & 0.4 & 0.2 \\
0.7  & 1.4 & 2.1 & 1.8 & 1.5 & 1.2 & 0.9 & 0.6 & 0.3 \\
0.6  & 1.2 & 1.8 & 2.4 & 2.0 & 1.6 & 1.2 & 0.8 & 0.4 \\
0.5  & 1.0 & 1.5 & 2.0 & 2.5 & 2.0 & 1.5 & 1.0 & 0.5 \\
0.4  & 0.8 & 1.2 & 1.6 & 2.0 & 2.4 & 1.8 & 1.2 & 0.6 \\
0.3  & 0.6 & 0.9 & 1.2 & 1.5 & 1.8 & 2.1 & 1.4 & 0.7 \\
0.2  & 0.4 & 0.6 & 0.8 & 1.0 & 1.2 & 1.4 & 1.6 & 0.8 \\
0.1  & 0.2 & 0.3 & 0.4 & 0.5 & 0.6 & 0.7 & 0.8 & 0.9 \\
\end{pmatrix}
\end{align*}

Matlab code for solving part a)

\begin{verbatim}
% Assignment 4 - problem a , finding LU and inverse of A of a
% finite difference matrix to solve T_xx = f(x), 0<=x<=1
% with T( x=0) = 0 , T( x=1) = 2
clear all
N = 10;

% Find the points of discretization
h = 1/N;
% Interval size of discretization
x = 0:h:1; % x = [x_0, x_1, ..., x_N]
x = x(2:end-1); % x = [x_1, x_2, ..., x_{N-1}]
% Construct the tri-diagonal matrix A
A = diag(ones(N-2,1),1) - 2*eye(N-1) + diag(ones(N-2,1),-1);

% Instead of the line above, use the line below to take advantage of the sparsity of A.
% We can construct A as a sparse matrix directly (this will save us some memory space
% and speed up computation)
% A = spdiags(ones(N-1,2), [-1;1], N-1, N-1) - 2*speye(N-1);
% Note: to answer this question, we don't need to setup the RHS, b

[L, U, P ] = lu(A);
sym(L)
sym(U)
A_inv = inv(A)

\end{verbatim}

 


\item
(10 pts) Compute the determinants of $L$ and $U$ for $N = 1000$ using \texttt{Matlab}'s determinant command. Why does the fact that they are both nonzero imply that $A$ is non-singular? How could you have computed these determinants really quickly yourself without \texttt{Matlab}'s determinant command?

\textbf{Solution:}

Determinants can be computed using Matlab command \texttt{det}. The determinant of $L$ is 1 and that of $U$ is -1000. The same answer could have
been obtained by hand by simply multiplying the diagonal elements of
the triangular matrices. 

The fact that the determinants of $L$ and $U$ are both nonzero implies that
A is nonsingular since,

$|A| = |L||U| = 0$


\end{enumerate}
\end{homeworkProblem}

  
% Problem 2
\begin{homeworkProblem} %(15)
\begin{enumerate}[a.]
\item
\begin{enumerate}[(i)]
\item
(10 pts) Use \texttt{Matlab} command \texttt{A=rand(4)} to generate a random 4-by-4 matrix and then use the function \texttt{qr} to find an orthogonal matrix $Q$ and an upper triangular matrix $R$ such that $A = QR$. Compute the determinants of $A$, $Q$ and $R$.

\textbf{Solution:} 
\begin{verbatim}
A = rand(4);
[Q, R]= qr(A);
det(A)
det(Q)
det(R)
\end{verbatim}

\item
(15 pts) Set \texttt{A=rand(n)} for at least 5 different $n$'s in \texttt{Matlab} for computing the determinant of $Q$ where $Q$ is the orthogonal matrix generated by \texttt{qr(A)}. What do you observe about the determinants of the matrices $Q$? Show, with a mathematical proof, that the determinant of any orthogonal matrix is either 1 or -1.

\textbf{Solution:} 

$Q^T Q = I  \Rightarrow  |QT Q| = 1  \Rightarrow |Q|^2 = 1$


\item
(15 pts) For a square $n \times n$ matrix matrix $B$, suppose there is an orthogonal matrix $Q$ and an upper-triangular matrix $R$ such that $B=QR$. Show that if a vector $x$ is a linear combination of the first $k$ column vectors of $B$ with $k\leq n$, then it can also be expressed as a linear combination of the first $k$ columns of $Q$.

\textbf{Solution:}

Let $\vec{b_i}$ , $\vec{q_i}$ be the i-th column of $B$ and $Q$ respectively. Then by expanding QR, we have

$$\vec{b_k} = \sum_{i=1}^{n} R_{ik} \vec{q_{i}} = \sum_{i=1}^k R_{ik} \vec{q_i}$$
The last equal sign holds because $R$ is an upper triangular matrix and
so $R_{ik} = 0$ for all $i > k$. Therefore, $\vec{b_j} \in$ span$\{\vec{q_1} , \vec{q_2} , \dots , \vec{q_k}\}$ for all
$j \leq k$. And so, $x \in$ span$\{\vec{b_1} , \vec{b_2} , \dots , \vec{b_k}\}   \Rightarrow x \in $ span$\{\vec{q_1}, \vec{q_2} , \dots , \vec{q_k}\}$.

\end{enumerate}



\item[b*.]
\begin{enumerate}[(i)]
\item 
(5 pts) Assume $\{\vec{v_1},\vec{v_2} \cdots \vec{v_n} \}$ is an orthonormal basis of $\mathbb{R}^n$. Suppose there exists a unit vector $\vec{u}$ such that $\vec{u}^T \vec{v_k} = 0$ for all $k=2,3\cdots n$, show that $\vec{u} = \vec{v_1}$ or $\vec{u} = -    \vec{v_1}$.

\textbf{Solution:}
Let $V$  be the matrix with $\vec{v_k}$ as columns. Then since the $\vec{v_k}$ are an orthonormal basis, we have $V^T V = I $  and $V\vec{\alpha} = \vec{u} $, for some vector $\vec{\alpha}$.    
$$V^T\vec{u} = \gamma \vec{e_1} \Rightarrow V^T V \vec{\alpha}= \gamma \vec{e_1}$$ Where $\vec{e_1}$ is the first elementary vector. Using the second equality, we have $\vec{\alpha} = \gamma \vec{e_1}$ and since $\vec{u}$ is a unit vector we must have $\gamma = 1$ or $\gamma = -1$. This implies $\vec{u} = \pm \vec{v_1}$\  

\item
(5 pts) Prove that if $C = QR$, where $Q$ is an orthogonal matrix and $R$ is an upper-triangular matrix with diagonal elements all positive, then the $Q$ and $R$ are unique.

\textbf{Solution:}
We want to show if there are any orthogonal matrix P and upper triangular matrix S with positive diagonal elements such that $P S = C$,
then $Q = P$ and $R = S$. Consider,
\begin{eqnarray*}
QR &=& PS\\
\Rightarrow P^TQ &=& SR\inv \mbox{  , an arbitrary upper triangular matrix}\\
\Rightarrow \vec{p_i}^T\vec{q_j} &=& 0 \; \mbox{    for all  } i < j\\
\mbox{Similarly...}\\
PS &=& QR\\
\Rightarrow Q^TP &=& RS\inv \mbox{  , an arbitrary upper triangular matrix}\\
\Rightarrow \vec{q_i}^T\vec{p_j} &=& 0 \; \mbox{    for all  } i < j\\
\end{eqnarray*}
So we have $\vec{p_i}^T\vec{q_j} = 0$ for all $i \ne j$ and $\vec{p_i}$ is a unit vector. So by the previous question, $\vec{p_i} = \pm \vec{q_i}$. Finally since $R_{ii} > 0, S_{ii} > 0$ for all $i$, and $(RS\inv)_{ii} = \frac{R_{ii}}{S_{ii}} > 0 \Rightarrow \vec{p_i} = \vec{q_i}$ which shows $P = Q$ and $R=S$.

Another way to see this let $M = Q^TP = RS\inv$ implies that $M$ is both orthogonal (since it is a product of orthogonal matrices) and upper triangular (since it is a product of upper triangular matrices). We therefore have that M must be the identity. That is, $\vec{m_1} = \vec{e_1} \Rightarrow (\vec{m_k})_1 = 0$ for all $k>1$. Therefore, $\vec{m_2} = \vec{e_2}$ which implies $(\vec{m_k})_2 = 0$ for all $k>2$ and so on which shows $M=I$.  

\end{enumerate}


\end{enumerate}
\end{homeworkProblem}
  
  
% Problem 3
\begin{homeworkProblem} %(15)
In class we have introduced the $LU$ decomposition of $A$, where $L$ is unit-lower triangular, in that it has ones along the diagonal, and $U$ is upper triangular. However, in the case of symmetric matrices, such as the discretization matrix, it is possible to decompose $A$ as $LDL^T$, where $L$ is still unit-lower triangular and $D$ is diagonal. This decomposition clearly shows off the symmetric nature
of $A$. 
\begin{enumerate}[a.]

\item
 (10 pts) Find the $LDL^T$ decomposition for
the matrix given in \textbf{Problem 1}. Show that $L$ is bidiagonal.
How do $D$ and $L$ relate to the matrix $U$ in the $LU$ decomposition of $A$? \\
\textbf{Hint:} Think about how $D$ and $L^T$ relate to $U$.\\
\textbf{Note:} Computing $LDL^T$ this way does not work out for any symmetric matrix, it only happens to work for this matrix in particular.

\textbf{Solution:} If we set $U = DL^T$, and if we keep in mind that $L^T$ is required to be unit-upper triangular (ones on the diagonal), then we can simply factor out the diagonal elements of $U$ as $D$ and what's left over will be $L^T$. In other words
\begin{align*}
U &= DL^T \\
L^T &= D^{-1}U\\
\end{align*}
$D$ can be obtained by the \texttt{Matlab} command \texttt{D = diag(diag(U))} and by doing \texttt{Lt = diag(1./diag(D))*U}. It's interesting that for this particular matrix it turns out that the result is nothing more than the transpose of the $L$ that was obtained by the \texttt{lu} command.


\item[b*.]
\begin{enumerate}[(i)]
\item
 (10 pts) To solve $A\vec{x} = \vec{b}$  we can exploit this new decomposition. We get $L D L^T \vec{x} = \vec{b}$
which we can now break into three parts: Solve $L\vec{y} = \vec{b}$ using forward substitution, now solve $D \vec{z} = \vec{y}$, and then solve $L^T \vec{x} = \vec{z}$ using back substitution. Write a \texttt{Matlab} code that does exactly this for arbitrary $N$ for the $A$ in \textbf{Problem 1}.

\textbf{Solution:}
The idea here is that we solve the system in a series of steps
\begin{align*}
Ax &= b \\
LDL^Tx &= b
\end{align*}
First solve this system using forward substitution (taking note that each equation in this system has at most two variables since $L$ is bi-diagonal)
\begin{align*}
Lz &= b
\end{align*}
Then solve this system by simply dividing by the diagonal element of $D$
\begin{align*}
Dy &= z
\end{align*}
Finally solve this system by backward substitution
\begin{align*}
L^Tx &= y
\end{align*}
Note that your code does not need to compute $L^T$ the same way as part a, you may directly use the transpose of the $L$ obtained from \texttt{lu}. You are also allowed to use \texttt{ldl} to get $L$ and $D$.
\item
(5 pts) Solve a system of the same form as \textbf{Problem 1} for $A$ of size $10$ and of size $1000$ with $\vec{b}$ having all zeros except 2 as the last entry in both cases, and verify the correctness of your solution using \texttt{Matlab}'s \texttt{A\textbackslash b} operator and the \texttt{norm} command.


\textbf{Solution:}
\begin{verbatim}

N = size(A, 1);
b = zeros(N, 1);
b(N) = 2; 

[L, U, P] = lu(A);
D = diag(diag(U));

% solve Ly=b
y = zeros(N, 1);
y(1) = b(1);
for i=2:N
    y(i) = b(i) - L(i, i-1) * y(i-1);
end

% solve Dz = y
z = y .* (1 ./ diag(D));

% solve L^Tx = z
Lt = L'; % for convenience
x = zeros(N, 1);
x(N) = z(N);
for i=(N-1):-1:1
  x(i) = z(i) - Lt(i, i+1) * x(i+1);
end

norm(A \ b - x)

\end{verbatim}
for both $N=10$ and $N=1000$ the norm difference is zero. 

\end{enumerate}
\end{enumerate}
\end{homeworkProblem}
\end{document}
 
 
 



